% Generated by GrindEQ Word-to-LaTeX 
\documentclass[10pt]{article} % use \documentstyle for old LaTeX compilers

\usepackage[english]{babel} % 'french', 'german', 'spanish', 'danish', etc.
\usepackage{amssymb}
\usepackage{amsmath}
\usepackage{txfonts}
\usepackage{mathdots}
\usepackage[classicReIm]{kpfonts}
\usepackage{graphicx}
\usepackage{ragged2e}

% You can include more LaTeX packages here 
\usepackage{setspace}

\begin{document}

%\selectlanguage{english} % remove comment delimiter ('%') and select language if required


\noindent 

\noindent 

\noindent 

\noindent 

\noindent 

\noindent 

\subsection*{}
\begin{Center}
{\Large Landmark Recognition}
\end{Center}        

\noindent 

\noindent 

\noindent 

\noindent 

\noindent 

\noindent 

\noindent 
\vspace{8mm} %5mm vertical space
\begin{Center}
       \textbf{\textit{\Large Bachelor of Technology}}\textit{}
\vspace{4mm}


\textit{\normalsize         in}
\vspace{5mm}


\textit{       }\textbf{\textit{\Large Computer Science \& Engineering}}

\vspace{6mm}
\noindent {\large Submitted by}
\vspace{4mm}
\noindent 

\noindent {\large Aditya Navin Nair}

\noindent {\large Ashwin MS}

\vspace{8mm}

\includegraphics*[width=2.82in, height=1.98in, keepaspectratio=false, trim=0.00in 0.00in 0.00in 0.02in]{image1}

\vspace{5mm}
\textbf{\Large Federal Institute of Science And Technology (FISAT){\circledR}}
\vspace{4mm}
{\large Angamaly, Ernakulam}

\noindent

\noindent {\large Affiliated to}

\vspace{2mm}

\textbf{\Large APJ Abdul Kalam Technological University}

\noindent \textbf{\Large CET Campus, Thiruvananthapuram}

\textbf{}

\textbf{}

\textbf{}
\end{Center}

\noindent 
\newpage

\begin{Center}
\centering FEDERAL INSTITUTE OF SCIENCE AND TECHNOLOGY

(FISAT)

Mookkannoor (P.O), Angamaly-683577


\vspace{5mm}


\noindent                   \includegraphics*[width=2.22in, height=1.55in, keepaspectratio=false, trim=0.00in 0.00in 0.00in 0.02in]{image2}

\vspace{10mm}
\centering         \textbf{\large CERTIFICATE}
\vspace{7mm}
\textbf{}

\justifying
\noindent 
This is to certify that the report entitled ``Landmark Recognition'' is a bonafide record of the mini project submitted by Ashwin M S(FIT19CS032), in partial fulfilment of the requirements for the award of the degree of Bachelor of Technology (B. Tech) in Computer Science \& Engineering during the academic year 2021-22.

\vspace{50mm} 
\end{Center}

\hspace{80mm}              \textbf{Dr. Jyothish K John}

Staff in Charge    \hspace{15mm}        Project Guide   \hspace{15mm}    Head of the Department 

\noindent 

\noindent 

\noindent 

\noindent 

\noindent 

\noindent 

\noindent 

\noindent 

\noindent 
\newpage
\begin{Center}

   

             \textbf{\Large Abstract}

\noindent \textbf{}

\noindent \textbf{}

\vspace{30mm}
\justifying
\begin{spacing}{1.5}
This project aims to create an application that recognizes various landmarks as well as its architecture from an image by implementing machine learning. This will allow users to easily recognize tourist attractions and attain information about these places. The challenge faced in this project is mainly the similarities between many of the structures in their architecture. Also we'll able to identify the landmark with a snippet of it's image. In the initial stage, we intend to classify landmarks with not many similarities and later on make classes that maybe similar in many ways.
\end{spacing}
\end{Center}
\newpage
\begin{Center}


\noindent \textbf{\Large Contribution by Author}

\vspace{20mm} 
\justifying
\begin{spacing}{1.5}
\noindent Training the model required acquiring a good dataset. Most datasets were not specific to the needs of the project. A dataset was created using images scraped from across the internet. The main contribution was towards fine tuning the AlexNet layers. The future of the project is in creating a CNN completely catered to this dataset.
\end{spacing}
\end{Center}
\newpage
\begin{Center}



            \textbf{\Large Acknowledgment}

\vspace{30mm}
 
\begin{spacing}{1.5}
\paragraph{}
\justifying
In the accomplishment of completion of my project on Landmark Recognition. I would like to convey my special gratitude to Mr. / Mrs. Pankaj Kumar G of Computer Science and Engineering Department for his kind support and guidance throughout the course of this project.
\paragraph{}
\justifying
I also express my sincere gratitude to the project coordinators for helping us throughout the course of this project, maintaining momentum and keeping track of all stages of the project.

\paragraph{}
\justifying
My sincere gratitude goes to Dr. Jyothish K John, HOD of Computer and Engineering Department for providing us the environment to complete the project in its stipulated time.Your valuable guidance and suggestions helped me in various phases of the completion of this project. I will always be thankful to you in this regard.
\paragraph{}
\justifying
Finally, I thank the university for including this project as a part of the curriculum, thereby giving me a real world project experience.
\paragraph{}
\end{spacing}
\justifying
I am ensuring that this project was finished by me and not copied.

\vspace{10mm}

\hspace{80mm} Aditya Navin Nair

\vspace{15mm} 

\hspace{95mm} Signature.
\end{Center}
\newpage


\noindent \textbf{}
\begin{Center}
\noindent \textbf{\Large Contents}

\end{Center}
\noindent \textbf{}

\noindent \textbf{}

\noindent \textbf{List of Figures                                             \hspace{110mm}...7}\newline
\noindent \textbf{1. Introduction                                             \hspace{109mm}...8}
\item \begin{enumerate}
\item \textbf{ }Overview{\dots}{\dots}{\dots}{\dots}{\dots}{\dots}{\dots}{\dots}{\dots}{\dots}{\dots}{\dots}{\dots}{\dots}{\dots}{\dots}{\dots}{\dots}{\dots}{\dots}{\dots}{\dots}{\dots}{\dots}{\dots}8

\item  Problem Statement{\dots}{\dots}{\dots}{\dots}{\dots}{\dots}{\dots}{\dots}{\dots}{\dots}{\dots}{\dots}{\dots}{\dots}{\dots}{\dots}{\dots}{\dots}{\dots}{\dots}{\dots}{\dots}.9

\item Objective{\dots}{\dots}{\dots}{\dots}{\dots}{\dots}{\dots}{\dots}{\dots}{\dots}{\dots}{\dots}{\dots}{\dots}{\dots}{\dots}{\dots}{\dots}{\dots}{\dots}{\dots}{\dots}{\dots}{\dots}{\dots}{\dots}9\textbf{}
\end{enumerate}


\noindent \textbf{}

\item \textbf{ 2. Design                                                  \hspace{115mm}...10}

\begin{enumerate}
\item \textbf{ }Implementation{\dots}{\dots}{\dots}{\dots}{\dots}{\dots}{\dots}{\dots}{\dots}{\dots}{\dots}{\dots}{\dots}{\dots}{\dots}{\dots}{\dots}{\dots}{\dots}{\dots}{\dots}{\dots}{\dots}.10

\item  Versions Proposed{\dots}{\dots}{\dots}{\dots}{\dots}{\dots}{\dots}{\dots}{\dots}{\dots}{\dots}{\dots}{\dots}{\dots}{\dots}{\dots}{\dots}{\dots}{\dots}{\dots}{\dots}{\dots}16
\end{enumerate}


\noindent 

\noindent
\textbf{3. Dataset \hspace{115mm}...17 \hspace{80mm}}

\begin{enumerate}
\item  Dataset Information{\dots}{\dots}{\dots}{\dots}{\dots}{\dots}{\dots}{\dots}{\dots}{\dots}{\dots}{\dots}{\dots}{\dots}{\dots}{\dots}{\dots}{\dots}{\dots}{\dots}{\dots}...17

\item  Dataset Acquisition{\dots}{\dots}{\dots}{\dots}{\dots}{\dots}{\dots}{\dots}{\dots}{\dots}{\dots}{\dots}{\dots}{\dots}{\dots}{\dots}{\dots}{\dots}{\dots}{\dots}{\dots}....18
\end{enumerate}
\noindent
\textbf{4. Source Code \hspace{110mm}...19}

\begin{enumerate}
\item  Training model{\dots}{\dots}{\dots}{\dots}{\dots}{\dots}{\dots}{\dots}{\dots}{\dots}{\dots}{\dots}{\dots}{\dots}{\dots}{\dots}{\dots}{\dots}{\dots}{\dots}{\dots}{\dots}.....19

\item   Image Scraper{\dots}{\dots}{\dots}{\dots}{\dots}{\dots}{\dots}{\dots}{\dots}{\dots}{\dots}{\dots}{\dots}{\dots}{\dots}{\dots}{\dots}{\dots}{\dots}{\dots}{\dots}{\dots}{\dots}..23\textbf{                                             }
\end{enumerate}
\noindent
\textbf{5. Output Images \hspace{105mm} ...32}

\vspace{10mm}
\noindent \textbf{}

\noindent \textbf{Bibliography \hspace{110mm} 44}


\newpage
\noindent \textbf{}
\begin{Center}
\textbf{\Large List of Figures}
\end{Center}
\noindent \textbf{}

\begin{enumerate}
\item \textbf{ }1.1: Some monuments{\dots} 8

\item  2.1: Architecture of LeNet{\dots} 10

\item  2.2: Architecture of AlexNet{\dots}15 

\item  3.1: Dataset distribution graph{\dots} 17

\item  5.1: Lotus Temple{\dots} 32

\item  5.2: India Gate{\dots} 33

\item  5.3(a): Indian Parliament{\dots} 34

\item  5.3(b): Indian Parliament{\dots} 34

\item  5.4(a): Mysore Palace{\dots} 35

\item  5.4(b): Mysore Palace{\dots} 35

\item  5.5: Rashtrapati Bhavan{\dots} 36

\item  5.6: Red Fort{\dots} 37

\item  5.7: Victoria Memorial{\dots} 38

\item  5.8(a): Taj Mahal{\dots} 39

\item  5.8(b): Taj Mahal{\dots} 39

\item  5.9: The Golden Temple Amritsar{\dots} 40

\item  5.10(a): Char Minar{\dots} 41

\item  5.10(b): Char Minar{\dots} 41

\item  5.11(a): Agra Fort{\dots} 42

\item  5.11(b): Agra Fort{\dots} 42

\item  5.12: Accuracy measurement on validation set{\dots} 43
\end{enumerate}


\newpage
\begin{Center}

\noindent \textbf{\Large Chapter 1}

\vspace{10mm}

\noindent \textbf{\Large Introduction}
\end{Center}
\noindent 

\noindent 

\noindent 

\begin{enumerate}
\item \begin{enumerate}
\item  \textbf{\large Overview }
\end{enumerate}
\end{enumerate}

\noindent \textbf{}
\paragraph{}
\justifying
\begin{spacing}{1.5} The development in electronic device like cameras, mobile phones and tablets with inbuilt camera as well their effective cost has made it to everyone hand. It is easier to take photos in the digital format. Most of the photos taken will travelling to tourist place are posted on social networking platforms on daily base. It has resulted in huge amount photos to available online. The touristic landmarks are easily recognizable of a well-known sites and architecture as in Fig.1.There are more than 1000 tourist places around the globe which have many architecture and monuments.  
 
\vspace{5mm}

\noindent \includegraphics*[width=4.5in, height=3.6in, keepaspectratio=false]{image3}

\noindent Figure 1.1: Some monuments
\paragraph{}
\justifying
\noindent Manually identifying the image in large scale is Time consuming and not traceable, therefore automatic content based solution is required. The main challenge is there no precise definition of what is and what not a landmark. With many Landmarks it is hard for single person to keep track of their detail all the time. With the vast amount of images over the internet which is easily be accessed.

\vspace{5mm}

\noindent \textbf{\large 1.2 Problem Statement}

\paragraph{}
\justifying
\noindent Often pictures of landmarks are seen popping up in websites and other sources without description or with unclear descriptions. This problem is mainly faced by students when they refer to the internet for projects and other study purposes. Another demographic which find this issue, are tourists. Often it can be confusing to find information about the places they visit, especially when they can't identify what or which monument they are visiting.~Another problem is to recognize what type of architecture style a landmark follows. E.g.: Mughal architecture. This problem has not been tackled by many machine learning models and will be useful for architecture enthusiasts to learn about different buildings.

\vspace{5mm}
\noindent \textbf{\large 1.3 Objective}

\paragraph{}
\justifying
\noindent Landmark and architecture Recognition is mainly used for architecture enthusiasts and school children to help them identify what type of architecture was used as well as to help identify which landmark it is.
\end{spacing}
\newpage 
\begin{Center}


\noindent \textbf{\Large Chapter 2}

\vspace{10mm}

\noindent \textbf{\Large Design}
\end{Center}
\textbf{}

\begin{enumerate}
\item \begin{enumerate}
\item \textbf{\large  Implementation }
\end{enumerate}
\end{enumerate}

\noindent \textbf{Models }
\vspace{5mm}
\noindent \textbf{}

\noindent \textbf{}

\noindent \textbf{2.1. LeNet}
\vspace{5mm}
\begin{spacing}{1.5}
\noindent \textbf{Structure of the LeNet network}

\noindent LeNet5 is a small network, it contains the basic modules of deep learning: convolutional layer, pooling layer, and full link layer. It is the basis of other deep learning models. Here we analyze LeNet5 in depth. At the same time, through example analysis, deepen the understanding of the convolutional layer and pooling layer.

\vspace{5mm}

\noindent 
\noindent \includegraphics*[width=4.4in, height=2.1in, keepaspectratio=false]{image5}

\vspace{2mm} 
\noindent Figure 2.1: Architecture of LeNet

\vspace{2mm} 

\noindent LeNet-5 Total seven layer , does not comprise an input, each containing a trainable parameters; each layer has a plurality of the Map the Feature , a characteristic of each of the input FeatureMap extracted by means of a convolution filter, and then each FeatureMap There are multiple neurons.
\paragraph{}
\vspace{3mm}

\vspace{2mm}
\paragraph{}
\textbf{INPUT LAYER}
\paragraph{}
\justifying
The first is the data INPUT layer. The size of the input image is uniformly normalized to 32 * 32.
\paragraph{}
\justifying
\noindent Note: This layer does not count as the network structure of LeNet-5. Traditionally, the input layer is not considered as one of the network hierarchy.
\noindent 
\paragraph{C1 layer-convolutional layer:}

\begin{enumerate}
\item \textbf{ Input picture}: 32 * 32

\item  \textbf{Convolution kernel size}: 5 * 5

\item  \textbf{Convolution kernel types}: 6

\item  \textbf{Output featuremap size}: 28 * 28 (32-5 + 1) = 28

\item  \textbf{Number of neurons}: 28~\textit{28~}6

\item  \textbf{Trainable parameters}: (5~\textit{5 + 1)~}6 (5 * 5 = 25 unit parameters and one bias parameter per filter, a total of 6 filters)

\item  \textbf{Number of connections}: (5~\textit{5 + 1)~}6~\textit{28~}28 = 122304
\end{enumerate}

\noindent 
\paragraph{S2 layer-pooling layer (downsampling layer):}

\begin{enumerate}
\item \textbf{ Input}: 28 * 28

\item  \textbf{Sampling area}: 2 * 2

\item  \textbf{Sampling method}: 4 inputs are added, multiplied by a trainable parameter, plus a trainable offset. Results via sigmoid

\item  \textbf{Sampling type}: 6

\item  \textbf{Output featureMap size}: 14 * 14 (28/2)

\item  \textbf{Number of neurons}: 14~\textit{14~}6

\item  \textbf{Trainable parameters}: 2 * 6 (the weight of the sum + the offset)

\item  \textbf{Number of connections}: (2~\textit{2 + 1)~}6~\textit{14~}14

\item  The size of each feature map in S2 is 1/4 of the size of the feature map in C1.
\end{enumerate}

\noindent 
\paragraph{C3 layer-convolutional layer:}

\begin{enumerate}
\item \textbf{ Input}: all 6 or several feature map combinations in S2

\item  \textbf{Convolution kernel size}: 5 * 5

\item  \textbf{Convolution kernel type}: 16

\item  \textbf{Output featureMap size}: 10 * 10 (14-5 + 1) = 10

\item  Each feature map in C3 is connected to all 6 or several feature maps in S2, indicating that the feature map of this layer is a different combination of the feature maps extracted from the previous layer.

\item  One way is that the first 6 feature maps of C3 take 3 adjacent feature map subsets in S2 as input. The next 6 feature maps take 4 subsets of neighboring feature maps in S2 as input. The next three take the non-adjacent 4 feature map subsets as input. The last one takes all the feature maps in S2 as input.

\item  \textbf{The trainable parameters are}: 6~\textit{(3~}5~\textit{5 + 1) + 6~}(4~\textit{5~}5 + 1) + 3~\textit{(4~}5~\textit{5 + 1) + 1~}(6~\textit{5~}5 +1) = 1516

\item  \textbf{Number of connections}: 10~\textit{10~}1516 = 151600
\end{enumerate}

\noindent 

\noindent 
\paragraph{S4 layer-pooling layer (down sampling layer):}

\begin{enumerate}
\item \textbf{ Input}: 10 * 10

\item  \textbf{Sampling area}: 2 * 2

\item  \textbf{Sampling method}: 4 inputs are added, multiplied by a trainable parameter, plus a trainable offset. Results via sigmoid

\item  \textbf{Sampling type}: 16

\item  \textbf{Output featureMap size}: 5 * 5 (10/2)

\item  \textbf{Number of neurons}: 5~\textit{5~}16 = 400

\item  \textbf{Trainable parameters}: 2 * 16 = 32 (the weight of the sum + the offset)

\item  \textbf{Number of connections}: 16~\textit{(2~}2 + 1)~\textit{5~}5 = 2000

\item  The size of each feature map in S4 is 1/4 of the size of the feature map in C3
\end{enumerate}

\noindent 

\noindent 
\paragraph{C5 layer-convolution layer :}

\begin{enumerate}
\item \textbf{ Input}: All 16 unit feature maps of the S4 layer (all connected to s4)

\item  \textbf{Convolution kernel size}: 5 * 5

\item  \textbf{Convolution kernel type}: 120

\item  \textbf{Output featureMap size}: 1 * 1 (5-5 + 1)

\item  \textbf{Trainable parameters / connection}: 120~\textit{(16~}5 * 5 + 1) = 48120
\end{enumerate}

\noindent 
\paragraph{F6 layer-fully connected layer :}

\begin{enumerate}
\item \textbf{ Input}: c5 120-dimensional vector

\item  \textbf{Calculation method}: calculate the dot product between the input vector and the weight vector, plus an offset, and the result is output through the sigmoid function.

\item  \textbf{Trainable parameters}: 84 * (120 + 1) = 10164
\end{enumerate}

\noindent 

\noindent 

\noindent 

\noindent 

\noindent 

\noindent 

\noindent 
\newpage
\noindent \textbf{2.2 AlexNet}
\vspace{2mm}
\paragraph{}
\justifying
\noindent AlexNet is an incredibly powerful model capable of achieving high accuracies on very challenging datasets. However, removing any of the convolutional layers will drastically degrade AlexNet's performance. AlexNet is a leading architecture for any object-detection task and may have huge applications in the computer vision sector of artificial intelligence problems. In the future, AlexNet may be adopted more than CNNs for image tasks.
\paragraph{}
\noindent As a milestone in making deep learning more widely-applicable, AlexNet can also be credited with bringing deep learning to adjacent fields such as natural language processing and medical image analysis.\textbf{}
\vspace{2mm}
\paragraph{}
\noindent \textbf{AlexNet Architecture}

\noindent \includegraphics*[width=5.90in, height=3.32in, keepaspectratio=false]{image6}

\noindent Figure 2.2: Architecture of AlexNet
\vspace{2mm}
\paragraph{}

\noindent AlexNet was the first~convolutional network~which used GPU to boost performance.~

\noindent 1.~AlexNet architecture consists of 5 convolutional layers, 3 max-pooling layers, 2 normalization layers, 2 fully connected layers, and 1 softmax layer.~

\noindent 2.~Each convolutional layer consists of convolutional filters and a nonlinear activation function ReLU.~

\noindent 3.~The pooling layers are used to perform max pooling.~

\noindent 4.~Input size is fixed due to the presence of fully connected layers.

\noindent 5.~The input size is mentioned at most of the places as 224x224x3 but due to some padding which happens it works out to be 227x227x3~

\noindent 6.~AlexNet overall has 60 million parameters.

\noindent 

\noindent 

\vspace{2mm}

\noindent \textbf{2.2 Versions proposed}
\paragraph{}
\noindent Version 0:

\noindent  Architecture -- LeNet is used as an initial stage of landmark recognition. This version is expected to have a lesser accuracy as compared to other CNNs.

\noindent  Libraries used - tensorflow, NumPy, Sklearn, Matplotlib
\paragraph{}
\noindent Version 1:

\noindent  Architecture -- Alexnet, a convolutional neural network is used where an input image example is given to the model in the form of tensor and used to identify the landmark and its architecture~

\noindent  Libraries used - tensorflow, NumPy, Sklearn, Matplotlib
\paragraph{}
\noindent Version 2(Not Completed):

\noindent  Architecture -- A CNN neural network which can classify the image with a better accuracy that version 1~

\noindent  Libraries used - tensorflow, NumPy, Sklearn, Matplotlib.

\noindent 

\noindent 

\noindent \textbf{}
\end{spacing}
\textbf{}
\newpage
\noindent \begin{center} \textbf{\Large Chapter 3}\end{center}
\paragraph{}
\begin{center} \textbf  {\Large Dataset} \end{center}

\textbf{}

\noindent \textbf{3.1. Dataset Information }

\noindent \textbf{}

\noindent \includegraphics*[width=6.25in, height=3.86in, keepaspectratio=false]{image7}

\noindent Figure 3.1: Dataset distribution graph

\begin{spacing}{1.5}
\noindent The dataset used comprises of images of 11 landmarks: Mysore Palace, Taj Mahal, The Golden Temple Amritsar, Red Fort, Agra Fort, Indian Parliament, Char Minar, India Gate, Rashtrapati Bhavan, Lotus Temple, Victoria Memorial

\noindent 

\noindent \begin{enumerate}
\item Mysore Palace 86~

\noindent \item Taj Mahal 82~

\noindent \item The Golden Temple Amritsar 84~

\noindent \item Red Fort 84~

\noindent \item Agra Fort 79~

\noindent \item Indian Parliament 75~

\noindent \item Char Minar 85~

\noindent \item India Gate 86~

\noindent \item Rashtrapati Bhavan 68~

\noindent \item Lotus Temple 93~

\noindent \item Victoria Memorial 76
\end{enumerate}

\noindent 

\noindent \textbf{3.2. Dataset Acquisition}
\vspace{2mm}
\noindent \textbf{}

\noindent \textbf{}

\noindent Acquiring a dataset for training the model, one by one, is a difficult task. Hence the images required were downloaded as batches using the program mentioned in Chapter 4 of this document.\textbf{}
\end{spacing}
\newpage





\begin{Center}

\noindent \textbf{\Large Chapter 4}

\end{Center}
\paragraph{}
\noindent \begin{center} \textbf{\large Source Code} \end{center}

\noindent 
\paragraph{}
\begin{spacing}{1.5}
\noindent \textbf{4.1 Training Model}

\vspace{3mm}

\noindent import~tensorflow~as~tf

\noindent from~google.colab~import~drive

\noindent drive.mount('/content/drive')

\noindent cd~drive/MyDrive

\noindent 

\noindent import~os,cv2

\noindent import~numpy~as~np

\noindent import~matplotlib.pyplot~as~plt

\noindent 

\noindent from~sklearn.utils~import~shuffle

\noindent from~sklearn.model\_selection~import~train\_test\_split

\noindent from~keras~import~backend~as~K

\noindent from~keras.utils~import~np\_utils

\noindent from~keras.models~import~Sequential

\noindent from~keras.layers.core~import~Dense,~Dropout,~Activation,~Flatten

\noindent from~keras.layers.convolutional~import~Convolution2D,~MaxPooling2D,~Conv2D

\noindent from~tensorflow.keras.optimizers~import~SGD,RMSprop

\noindent PATH~=~os.getcwd()

\noindent print(PATH)

\noindent \#~Define~data~path

\noindent data\_path~=~PATH~+~'/dataset-landmark'

\noindent data\_path1~=~PATH~+~'/dataset-landmark\_test'

\noindent data\_dir\_list~=~os.listdir(data\_path)

\noindent print(data\_dir\_list)

\noindent img\_rows=128

\noindent img\_cols=128

\noindent num\_channel=1

\noindent num\_epoch=20

\noindent 

\noindent 

\noindent dataimg\_train=tf.keras.utils.image\_dataset\_from\_directory(

\noindent ~~~~directory=data\_path,

\noindent ~~~~labels='inferred',

\noindent ~~~~label\_mode='categorical',

\noindent ~~~~class\_names=None,

\noindent ~~~~color\_mode='rgb',

\noindent ~~~~batch\_size=32,

\noindent ~~~~image\_size=(256,~256),

\noindent ~~~~shuffle=True,

\noindent ~~~~seed=1,

\noindent ~~~~validation\_split=None,

\noindent ~~~~subset=None,

\noindent ~~~~interpolation='bilinear',

\noindent ~~~~follow\_links=False,

\noindent ~~~~crop\_to\_aspect\_ratio=False,

\noindent )

\noindent dataimg\_valid=tf.keras.utils.image\_dataset\_from\_directory(

\noindent ~~~~directory=data\_path1,

\noindent ~~~~labels='inferred',

\noindent ~~~~label\_mode='categorical',

\noindent ~~~~class\_names=None,

\noindent ~~~~color\_mode='rgb',

\noindent ~~~~batch\_size=32,

\noindent ~~~~image\_size=(256,~256),

\noindent ~~~~shuffle=True,

\noindent ~~~~seed=1,

\noindent ~~~~validation\_split=None,

\noindent ~~~~subset=None,

\noindent ~~~~interpolation='bilinear',

\noindent ~~~~follow\_links=False,

\noindent ~~~~crop\_to\_aspect\_ratio=False,

\noindent )

\noindent 

\noindent 

\noindent \#Test list 
\justifying
\noindent test\_list = [ "/content/drive/MyDrive/Demo\_Images\_landmarks/LotusTemple.jpeg",\newline
"/content/drive/MyDrive/Demo\_Images\_landmarks/IndiaGate.jpeg",\newline "/content/drive/MyDrive/Demo\_Images\_landmarks/IndianParliament.jpeg",\newline
"/content/drive/MyDrive/Demo\_Images\_landmarks/IndianParliament1.jpeg",\newline
"/content/drive/MyDrive/Demo\_Images\_landmarks/MysorePalace.jpeg",\newline "/content/drive/MyDrive/Demo\_Images\_landmarks/MysorePalace1.jpeg",\newline "/content/drive/MyDrive/Demo\_Images\_landmarks/RashtrapatiBhavan.jpeg",\newline
"/content/drive/MyDrive/Demo\_Images\_landmarks/RashtrapatiBhavan1.jpeg",\newline "/content/drive/MyDrive/Demo\_Images\_landmarks/RedFort.jpeg",\newline "/content/drive/MyDrive/Demo\_Images\_landmarks/VictoriaMemorial.jpeg" ]

\noindent 

\noindent \#CNN~Architecture

\noindent model~=~Sequential()

\noindent model.add(tf.keras.layers.Conv2D(6,~kernel\_size=(5,~5),~activation='relu',~input\_shape=(256,~256,~3)))

\noindent model.add(tf.keras.layers.MaxPooling2D(pool\_size=(2,~2)))

\noindent model.add(tf.keras.layers.Conv2D(16,~kernel\_size=(5,~5),~activation='relu'))

\noindent model.add(tf.keras.layers.Conv2D(384,~kernel\_size=(3,3),~strides=~1,

\noindent ~~~~~~~~~~~~~~~~~~~~padding=~'same',~activation=~'relu',

\noindent ~~~~~~~~~~~~~~~~~~~~kernel\_initializer=~'he\_normal'))

\noindent 

\noindent model.add(tf.keras.layers.Conv2D(256,~kernel\_size=(3,3),~strides=~1,

\noindent ~~~~~~~~~~~~~~~~~~~~padding=~'same',~activation=~'relu',

\noindent ~~~~~~~~~~~~~~~~~~~~kernel\_initializer=~'he\_normal'))

\noindent model.add(tf.keras.layers.MaxPooling2D(pool\_size=(2,~2)))

\noindent model.add(Dropout(0.5))

\noindent model.add(tf.keras.layers.Flatten())

\noindent model.add(tf.keras.layers.Dense(120,~activation='relu'))

\noindent model.add(tf.keras.layers.Dense(11,~activation='softmax'))

\noindent 

\noindent model.compile(loss=tf.keras.metrics.categorical\_crossentropy,~optimizer=tf.keras.optimizers.Adam(),~metrics=['accuracy'])

\noindent model.summary()

\noindent model.fit(dataimg\_train,~epochs=15)

\noindent model.save("savedweights\_final")

\noindent from~keras.models~import~load\_model

\noindent model~=load\_model("savedweights\_final")

\noindent score~=~model.evaluate(dataimg\_valid)

\noindent print('Test~Loss:',~score[0])

\noindent print('Test~accuracy:',~score[1])

\noindent from~keras.models~import~load\_model

\noindent from~keras.preprocessing~import~image

\noindent import~matplotlib.pyplot~as~plt

\noindent import~numpy~as~np

\noindent import~os

\noindent 

\noindent def~load\_image(img\_path,~show=False):

\noindent 

\noindent ~~~~img~=~image.load\_img(img\_path,~target\_size=(256,~256))

\noindent ~~~~img\_tensor~=~image.img\_to\_array(img)~~~~~~~~~~~~~~~~~~~~\#~(height,~width,~channels)

\noindent ~~~~img\_tensor~=~np.expand\_dims(img\_tensor,~axis=0)~~~~~~~~~\#~(1,~height,~width,~channels),~add~a~dimension~because~the~model~expects~this~shape:~(batch\_size,~height,~width,~channels)

\noindent ~~~~img\_tensor~/=~255.~~~~~~~~~~~~~~~~~~~~~~~~~~~~~~~~~~~~~

\noindent ~\#~imshow~expects~values~in~the~range~[0,~1]

\noindent 

\noindent ~~~~if~show:

\noindent ~~~~~~~~plt.imshow(img\_tensor[0])~~~~~~~~~~~~~~~~~~~~~~~~~~~

\noindent ~~~~~~~~plt.axis('off')

\noindent ~~~~~~~~plt.show()

\noindent 

\noindent ~~~~return~img\_tensor

\noindent 

\noindent if~\_\_name\_\_~==~"\_\_main\_\_":

\noindent 

\noindent ~~~~\#~load~model

\noindent ~~~~\#model~=~load\_model("savedweights\_final")

\noindent 

\noindent ~~~~\#~image~path

\noindent ~~~~img\_path~=~test\_list[0]

\noindent ~~~~\#~load~a~single~image

\noindent ~~~~new\_image~=~load\_image(img\_path)

\noindent 

\noindent ~~~~\#~check~prediction

\noindent ~~~~pred~=~model.predict(new\_image)

\noindent ~~~~img=plt.imread(img\_path)

\noindent ~~~~plt.imshow(img)

\noindent ~~~~print(dataimg\_train.class\_names[pred.argmax()])

\vspace{2mm}
\noindent \textbf{\large 4.2 Image Scraper}

\vspace{2mm}
\noindent \textbf{4.2.1 main.py}

\noindent \# Import libraries\newline
import os\newline
import concurrent.futuresfrom GoogleImageScraper
\newline 
import GoogleImageScraperfrom patch \newline
import webdriver\_executable\newline
defworker\_thread(search\_key):\newline
image\_scraper = GoogleImageScraper(    \newline
webdriver\_path, image\_path, search\_key, number\_of\_images, headless,\newline min\_resolution, max\_resolution)  \newline 
image\_urls = image\_scraper.find\_image\_urls()\newline    image\_scraper.save\_images(image\_urls) \newline
\# Release resources \newline
del image\_scraper \newline 
if \_\_name\_\_ == "\_\_main\_\_":    
\# Define file path    \newline
webdriver\_path = os.path.normpath(os.path.join(os.getcwd(),'webdriver',webdriver\_executable()))  \newline
image\_path = os.path.normpath(os.path.join(os.getcwd(), 'photos'))  \newline
\# Add new search key into array ["cat","t-shirt","apple","orange","pear","fish"]  \newline
search\_keys = ["Indian Parliament Building"]  \newline
\# Parameters  \newline
number\_of\_images = 500     \hspace{2mm}    \# Desired number of images\newline headless = True    \hspace{2mm}    \# True = No Chrome GUI \newline   min\_resolution = (0, 0)           \# Minimum desired image resolution\newline max\_resolution = (9999, 9999)     \# Maximum desired image resolution\newline max\_missed = 1000              \# Max number of failed images before exit\newline    number\_of\_workers = 1       \# Number of "workers" used \newline    \# Run each search\_key in a separate thread \newline
\# Automatically waits for all threads to finish    with concurrent.futures.ThreadPoolExecutor(max\_workers=\newline number\_of\_workers) as executor: \newline       executor.map(worker\_thread, search\_keys)

\noindent 

\vspace{2mm}

\noindent \textbf{\large 4.2.2 GoogleImageScraper.py}
\vspace{2mm}

\noindent \textbf{}

\noindent \#import selenium driversfrom selenium \newline
import webdriverfrom selenium.webdriver.chrome.options \newline import Options
\newline\#import helper librariesimport timeimport urllib.request\newline 
import os\newline import requests\newline import io from PIL \newline import Image\newline
\#custom patch libraries\newline
import patch \newline
class GoogleImageScraper():
\newline def \_\_init\_\_(self, webdriver\_path, image\_path, search\_key="cat", number\_of\_images=1, headless=True, min\_resolution=(0, 0), max\_resolution=(1920, 1080), max\_missed=10):    \newline    \#check parameter types     \newline   image\_path = os.path.join(image\_path, search\_key)        \newline if (type(number\_of\_images)!=int):    \newline        print("[Error] Number of images must be integer value.")     \newline       return   \newline     if not os.path.exists(image\_path):   \newline
print("[INFO] Image path not found. Creating a new folder.")    \newline        os.makedirs(image\_path)    \newline
\#check if chromedriver is updated  \newline
while(True):    \newline
try:    \newline
\#try going to www.google.com       \newline
options = Options()         \newline       
if(headless):       \newline
options.add\_argument('--headless')         \newline
driver = webdriver.Chrome(webdriver\_path, chrome\_options=options) 
\newline 
driver.set\_window\_size(1400,1050)               \newline driver.get("https://www.google.com")     \newline
break       \newline
except:         \newline
\#patch chromedriver if not available or outdated       \newline
try:            \newline
driver      \newline
except NameError:           \newline
is\_patched = patch.download\_lastest\_chromedriver()       \newline
else:            \newline
is\_patched = patch.download\_lastest\_chromedriver(driver.capabilities['version'])         \newline
if (not is\_patched):           \newline
exit("[ERR] Please update the chromedriver.exe in the webdriver folder according to your chrome version:https://chromedriver.chromium.org/downloads")            \newline
self.driver = driver    \newline
self.search\_key = search\_key      \newline
self.number\_of\_images = number\_of\_images    \newline
self.webdriver\_path = webdriver\_path      \newline
self.image\_path = image\_path      \newline
self.url = "https://www.google.com/search?q=\%s\&source=lnms\&tbm=isch\&sa=X\&ved=2ahUKEwie44\_AnqLpAhUhBWMBHUFGD90Q\_AUoAXoECBUQAw\&biw=1920\&bih=947"\%(search\_key)  \newline
self.headless=headless  \newline
self.min\_resolution = min\_resolution  \newline
self.max\_resolution = max\_resolution  \newline
self.max\_missed = max\_missed      \newline
def find\_image\_urls(self):\textit{        }print("[INFO] Gathering image links")     \newline
image\_urls=[]  \newline
count = 0   \newline    
missed\_count = 0  \newline
self.driver.get(self.url)   \newline
time.sleep\eqref{GrindEQ__3_}   \newline
indx = 1    \newline
while self.number\_of\_images $\mathrm{>}$ count:   \newline
try:        \newline
\#find and click image          \newline
imgurl = self.driver.find\_element\_by\_xpath('//*[@id="islrg"]/div[1]/div[\%s]/a[1]/div[1]/img'\%(str(indx)))      \newline
imgurl.click() \newline
missed\_count = 0  \newline
except Exception:  \newline
\#print("[-] Unable to click this photo.")      \newline
missed\_count = missed\_count + 1   \newline
if(missed\_count$\mathrm{>}$self.max\_missed):  \newline
print("[INFO] Maximum missed photos reached, exiting...")  \newline
break \newline
try:    \newline
\#select image from the popup  \newline
time.sleep\eqref{GrindEQ__1_} \newline
class\_names = ["n3VNCb"] \newline
images = [self.driver.find\_elements\_by\_class\_name(class\_name)
\newline for class\_name in class\_names if\newline len(self.driver.find\_elements\_by\_class\_name(class\_name)) != 0 ][0] \newline
for image in images:    \newline
\#only download images that starts with http    
\newline
src\_link = image.get\_attribute("src")         \newline
if(("http" in  src\_link) and (not "encrypted" in src\_link)):      \newline
print(f"[INFO] $\mathrm{\{}$self.search\_key$\mathrm{\}}$ {\textbackslash}t \#$\mathrm{\{}$count$\mathrm{\}}$ {\textbackslash}t $\mathrm{\{}$src\_link$\mathrm{\}}$")       \newline
image\_urls.append(src\_link)  \newline
count +=1 \newline
break   \newline
except Exception: \newline
print("[INFO] Unable to get link")      \newline
try:            \newline
\#scroll page to load next image        \newline
if(count\%3==0):    \newline
self.driver.execute\_script("window.scrollTo(0, "+str(indx*60)+");")        \newline
element = self.driver.find\_element\_by\_class\_name("mye4qd")               \newline
element.click()    \newline
print("[INFO] Loading next page")            \newline
time.sleep\eqref{GrindEQ__3_}       \newline
except Exception:      \newline
time.sleep\eqref{GrindEQ__1_}            indx += 1      \newline
self.driver.quit()  \newline
print("[INFO] Google search ended")   \newline   
return image\_urls  \newline
def save\_images(self,image\_urls):      \newline
\#save images into file directory\textit{        }\newline 
print("[INFO] Saving image, please wait...")    \newline
for indx,image\_url in enumerate(image\_urls):      \newline
try:    \newline
print("[INFO] Image url:\%s"\%(image\_url))     \newline
search\_string = ''.join(e for e in self.search\_key if e.isalnum())\newline
image = requests.get(image\_url,timeout=5)      \newline
if image.status\_code == 200:       \newline
with Image.open(io.BytesIO(image.content)) as image\_from\_web: \newline
try:        \newline
filename = "\%s\%s.\%s"\%(search\_string,str(indx),image\_from\_web.format.lower()) \newline
image\_path = os.path.join(self.image\_path, filename)                      \newline
print(f"[INFO] $\mathrm{\{}$self.search\_key$\mathrm{\}}$ {\textbackslash}t $\mathrm{\{}$indx$\mathrm{\}}$ {\textbackslash}t Image saved at: $\mathrm{\{}$image\_path$\mathrm{\}}$")    \newline                        image\_from\_web.save(image\_path)  \newline
except OSError:     \newline
rgb\_im = image\_from\_web.convert('RGB')   \newline
rgb\_im.save(image\_path)   \newline
image\_resolution = image\_from\_web.size   \newline
if image\_resolution != None:   \newline
if image\_resolution[0]$\mathrm{<}$self.min\_resolution[0] or\newline
image\_resolution[1]$\mathrm{<}$self.min\_resolution[1] or\newline
image\_resolution[0]$\mathrm{>}$self.max\_resolution[0] or\newline
image\_resolution[1]$\mathrm{>}$self.max\_resolution[1]:\newline
image\_from\_web.close()\newline
\#print("GoogleImageScraper Notification: \%s did not meet resolution requirements."\%(image\_url))                              \newline
os.remove(image\_path)  \newline
image\_from\_web.close()    \newline
except Exception as e:  \newline
print("[ERROR] Download failed: ",e)    \newline
pass    \newline
print("--------------------------------------------------") \newline
print("[INFO] Downloads completed. Please note that some photos were not downloaded as they were not in the correct format (e.g. jpg, jpeg, png)")\newline


\noindent 

\noindent \textbf{}

\noindent \textbf{4.2.3 patch.py}

\noindent \textbf{}

\noindent \#!/usr/bin/env python3\newline
from selenium import webdriver\newline from selenium.webdriver.common.keys import Keys\newline from selenium.common.exceptions import WebDriverException, SessionNotCreatedException\newline import sys\newline import os\newline import urllib.requestimport reimport zipfileimport stat\newline from sys import platform\newline def webdriver\_executable():    \newline if platform == "linux" or platform == "linux2" or platform == "darwin":        \newline return 'chromedriver'    \newline return 'chromedriver.exe'\newline def download\_lastest\_chromedriver(current\_chrome\_version=""):    \newline def get\_platform\_filename():\newline        filename = ''\newline        is\_64bits = sys.maxsize $\mathrm{>}$ 2**32\newline            if platform == "linux" or platform == "linux2":\newline            \# linux            filename += 'linux'\newline            filename += '64' if is\_64bits else '32'\newline        elif platform == "darwin":\newline            \# OS X\newline            filename += 'mac64'\newline        elif platform == "win32":\newline            \# Windows...\newline            filename += 'win32'\newline            filename += '.zip'\newline            return filename\newline        \# Find the latest chromedriver, download, unzip, set permissions to executable.\newline        result = False\newline    try:\newline        url = 'https://chromedriver.chromium.org/downloads'\newline        base\_driver\_url = 'https://chromedriver.storage.googleapis.com/'\newline        file\_name = 'chromedriver\_' + get\_platform\_filename()\newline        pattern = 'https://.*?path=({\textbackslash}d+{\textbackslash}.{\textbackslash}d+{\textbackslash}.{\textbackslash}d+{\textbackslash}.{\textbackslash}d+)'\newline            \# Download latest chromedriver.\newline        stream = urllib.request.urlopen(url)\newline        content = stream.read().decode('utf8')\newline            \# Parse the latest version.\newline        all\_match = re.findall(pattern, content)\newline                if all\_match:\newline            \# Version of latest driver.\newline            if(current\_chrome\_version!=""):\newline                print("[INFO] updating chromedriver")\newline                all\_match = list(set(re.findall(pattern, content)))\newline                current\_chrome\_version = ".".join(current\_chrome\_version.split(".")[:-1])\newline                version\_match = [i for i in all\_match if re.search("$\mathrm{\wedge}$\%s"\%current\_chrome\_version,i)]\newline                version = version\_match[0]\newline            else:                print("[INFO] installing new chromedriver")\newline                version = all\_match[1]\newline            driver\_url = base\_driver\_url + version + '/' + file\_name\newline                \# Download the file.\newline            print('[INFO] downloading chromedriver ver: \%s: \%s'\% (version, driver\_url))\newline            app\_path = os.path.dirname(os.path.realpath(\_\_file\_\_))\newline            chromedriver\_path = os.path.normpath(os.path.join(app\_path, 'webdriver', webdriver\_executable()))\newline            file\_path = os.path.normpath(os.path.join(app\_path, 'webdriver', file\_name))\newline            urllib.request.urlretrieve(driver\_url, file\_path) \newline               \# Unzip the file into folder\newline            with zipfile.ZipFile(file\_path, 'r') as zip\_ref:\newline                zip\_ref.extractall(os.path.normpath(os.path.join(app\_path, 'webdriver')))\newline            st = os.stat(chromedriver\_path)\newline            os.chmod(chromedriver\_path, st.st\_mode {\textbar} stat.S\_IEXEC)\newline            print('[INFO] lastest chromedriver downloaded')\newline            \# Cleanup.\newline            os.remove(file\_path)\newline            result = True\newline    except Exception:\newline        print("[WARN] unable to download lastest chromedriver. the system will use the local version instead.")\newline        return result\newline

\end{spacing}
\newpage
 
\noindent  \begin{center}\textbf {Chapter 5} \end{center}

\noindent \textbf{}

\noindent \textbf{}

\noindent \begin{center} \textbf{Output Images}\end{center}

\noindent \textbf{}

\noindent 

\noindent \textbf{5.1 Lotus Temple}

\noindent \textbf{}

\noindent \includegraphics*[width=3.53in, height=3.25in, keepaspectratio=false]{image8}

\noindent 

\noindent Figure 5.1: Lotus Temple

\noindent 

\noindent 
\newpage
\noindent \textbf{5.2 India Gate}
\newline
\newline
\noindent \includegraphics*[width=3.09in, height=3.37in, keepaspectratio=false]{image9}\textbf{ } \newline Figure 5.2: India Gate

\noindent \textbf{}
\newpage
\noindent \textbf{5.3 Indian Parliament}

\noindent 

\noindent 

\noindent \includegraphics*[width=3.49in, height=3.61in, keepaspectratio=false]{image10}

\noindent Fig 5.3(a): Indian Parliament (top), Fig 5.3(b): Indian Parliament (below)

\noindent \includegraphics*[width=3.49in, height=3.5in, keepaspectratio=false]{image11}

\noindent 

\noindent 

\noindent \textbf{5.4 Mysore Palace}

\newline

\noindent \includegraphics*[width=3.3in, height=3.5in, keepaspectratio=false]{image12}

\noindent Figure 5.4(a): Mysore Palace

\noindent 

\noindent \includegraphics*[width=3.3in, height=3.5in, keepaspectratio=false]{image13}

\noindent Figure 5.4(b): Mysore Palace

\noindent 

\newpage 
\noindent \textbf {5.5 Rashtrapati Bhavan}

\newline \includegraphics*[width=4.16in, height=4.20in, keepaspectratio=false]{image14}

\noindent Figure 5.5: Rashtrapati Bhavan

\newpage
\textbf{5.6 Red Fort}\newline

 \includegraphics*[width=4.10in, height=3.97in, keepaspectratio=false]{image15}

\noindent Figure 5.6: Red Fort

\noindent 

\newpage \textbf{5.7 Victoria Memorial} \newline


\noindent \includegraphics*[width=4.10in, height=3.55in, keepaspectratio=false]{image16}

\noindent Figure 5.7: Victoria Memorial \newline

\newpage
\textbf{5.8 Taj Mahal}

\includegraphics*[width=3.5in, height=3.1in, keepaspectratio=false]{image17}

\noindent Figure 5.8(a): Taj Mahal\newline
\includegraphics*[width=3.5in, height=3.1in, keepaspectratio=false]{image18}

\noindent 

\noindent Figure 5.8(b): Taj Mahal \newline

\newpage
\noindent \textbf{5.9 The Golden Temple Amritsar}\newline

\noindent \textbf{}

\noindent \includegraphics*[width=5.21in, height=4.20in, keepaspectratio=false]{image19}

\noindent Figure 5.9: The Golden Temple Amritsar

\newpage
\noindent \textbf{5.10 Char Minar}\newline

\noindent 

\noindent \includegraphics*[width=3.5in, height=3.07in, keepaspectratio=false]{image20}

\noindent Figure 5.10(a): Char Minar

\noindent \includegraphics*[width=3.5in, height=3.07in, keepaspectratio=false]{image21}

\noindent Figure 5.10(b): Char Minar \newline

\noindent \textbf{}

\newpage
\noindent \textbf{5.11 Agra Fort} \newline

\noindent \includegraphics*[width=3.5in, height=2.27in, keepaspectratio=false]{image22}

\noindent Figure 5.11(a): Agra Fort

\noindent 

\noindent \includegraphics*[width=3.75in, height=3.53in, keepaspectratio=false]{image23}

\noindent Figure 5.11(b): Agra Fort

\newpage

\noindent \textbf{5.12 Accuracy}

\noindent \textbf{}

\noindent \textbf{\includegraphics*[width=5.75in, height=1.57in, keepaspectratio=false]{image24}}

\noindent 

\noindent Figure 5.12: Accuracy measurement on validation set

\newpage
\noindent \begin{center}
    \textbf{\Large Bibliography}
\end{center} 

\noindent \textbf{}

\noindent 
{1. ImageNet Classification with Deep Convolutional Neural Networks- \newline https://papers.nips.cc/paper/2012/hash/c399862d3b9d6b76c8436e924a68c45b-Abstract.html} \newline

\noindent 2. Dive Into Deep Learning, d2l.ai


\end{document}

